\documentclass{beamer}  
\usepackage{amsmath}
\usepackage{graphicx}
\usepackage{url}
\usepackage{color}

\makeatletter
\def\url@smallurlstyle{%
 \@ifundefined{selectfont}{\def\UrlFont{\sf}}{\def\UrlFont{\footnotesize\ttfamily}}}
\makeatother
\urlstyle{smallurl}

\mode<presentation>
{ \usetheme{Darmstadt} }

\title{OpenEmbedded, Past, Present, and Future}

\institute{Open SDR}
\author{Philip Balister \\
\tt\tiny philip@opensdr.com}

\date{February 3, 2020}
 
\begin{document} 

\begin{frame}
\titlepage
\end{frame}

\section*{Outline}

\begin{frame}
  \tableofcontents
\end{frame}

\section{Introduction}

\begin{frame}
\frametitle{What is OpenEmbedded}

\begin{itemize}
\item Started as a build system for embedded devices
\item Three basic principles
	\begin{itemize}
		\item{Machine}
		\item{Distribution}
		\item{Image}
	\end{itemize}
\end{itemize}

\end{frame}

\begin{frame}
\frametitle{What is the Yocto Project}

\begin{itemize}
\item Linux Foundation's first collaborative project
\item Membership is limited to Linux Foundation members
\item Three membership level Silver, Gold and Platinum
\item Provides financial support for testing hardware
\item Pays Richard Purdie and Michael Halstead
\end{itemize}

\end{frame}

\section{Past}

\begin{frame}
\frametitle{OpenEmbedded, the early days}

%\begin{center}
%\includegraphics[width=3.0in]{omap3_block_diagram.png}
%\end{center}

	\begin{itemize}
		\item This is before my time, so don't be angry if I get this wrong!
		\item Created in 2003 by Chris Larson, Michael Lauer, and Holger Schurig
		\item Unification of OpenZaurus, Familiar Linux and OpenSIMpad projects
	\end{itemize}

\end{frame}

\begin{frame}
\frametitle{OpenEmbedded, things get serious! }

\begin{itemize}
\item Richard Purdie create Poky repo at OpenedHand in 2005
\item First OpenEmbedded Stand at FOSDEM 2007
\item FOSDEM 2008 OpenEmbedded forms an organization to support the project
\item Monta Vista bases Linux product on OpenEmbedded
\end{itemize}

\end{frame}

\begin{frame}
	\frametitle{OpenEmbedded Developers}

\begin{center}
\includegraphics[width=2.0in]{RP-etal.png}
\end{center}

\end{frame}


\begin{frame}
	\frametitle{Excitement!}

\begin{center}
\includegraphics[width=3.0in]{Excitement.png}
\end{center}

\end{frame}
\begin{frame}

	\frametitle{OpenEmbedded and the Yocto Project}

\begin{itemize}
\item In 2010, the Yocto Project is announced at ELCE in Cambridge, UK
\begin{itemize}
	\item The first of many Linux Foundation Collaborative Projects
\end{itemize}
\item OpenEmbedded is the Yocto Project technical partner
\item OpenEmbedded has a representative on Yocto Project governing board.
\item Both organizations have Technical Steering committees
\item OpenEmbedded represents the people involved with the project
\item The Yocto Project provides an organizations for corporate entities to join
\item Confusing? Yes it is.
\end{itemize}

\end{frame}

\begin{frame}
\frametitle{What really changed}

\begin{itemize}
	\item OpenEmbedded Classic had one layer for all recipe.
		\begin{itemize}
			\item Yes I know about collections
		\end{itemize}
	\item Many developers had push access
	\item Uneven testing
	\item No concept of a release
	\item Everyone was frustrated
\end{itemize}

\end{frame}

\begin{frame}
\frametitle{Process Improvements}

\begin{itemize}
	\item OpenEmbedded-core created with a subset of recipes
	\item Layer concept extendeded to collect recipes into subsets
	\item OpenEmbedded-core tests patches before the merge
	\item Builds for several ARM and x84 qemu machines
	\item Two releases per year
\end{itemize}

\end{frame}


\section{Present}

\begin{frame}
\frametitle{The Good}

\begin{itemize}
\item Core layers build consistently
\item Healthy job market for developers
\item Layers are available for broad spectrum of software
\item You can do anything you want
\end{itemize}

\end{frame}

\begin{frame}
\frametitle{The Bad}

\begin{itemize}
\item You can do anything you want
\item Error reporting can be confusing
\item Usability issues persist
\end{itemize}

\end{frame}

\section{Future}

\begin{frame}
\frametitle{The next ten years!}

\begin{itemize}
\item What features do we need?
\item How do we get new people involved?
\item How do we communicate with end users?
\end{itemize}

\end{frame}

\begin{frame}
\frametitle{Questions}

Answer my Questions!

\end{frame}

\end{document}
